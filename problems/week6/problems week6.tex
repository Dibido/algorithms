\documentclass[]{article}
\usepackage{mathtools}
\usepackage{listings}
\usepackage{enumitem}
\usepackage{amsmath}
\usepackage{amsfonts}
\usepackage{amssymb}
\usepackage[T1]{fontenc} %use different encoding (copy from pdf is now possible}
\usepackage{fullpage} %small margins
\usepackage{color}
\usepackage{graphicx}
\usepackage{float}
\definecolor{light-gray}{gray}{0.95}
\DeclarePairedDelimiter\floor{\lfloor}{\rfloor}
\lstset{
	numbers=left,
	breaklines=true,
	backgroundcolor=\color{light-gray},
	tabsize=4,
	literate={\ \ }{{\ }}1
}

%opening
\title{Problem session 6}
\author{Dibran Dokter 1047390 \& Marnix Lukasse 1047400}

\begin{document}
	
	\maketitle
	
	\section*{6}
	
	\subsection*{6.1}
	
	1)
	
	\begin{figure}[H]
		\includegraphics[scale=0.3]{/home/owner/git/algorithms/problems/week6/images/experimentgraph.jpeg}
	\end{figure}

	2)
	
	To find the best combination of experiments we can look at the difference between the cost and the profit for every combination of the experiments.\\
	For example, for combination $E_1$ and $E_2$ we sum the total payout, look what tools are needed and sum the total cost. Profit = sum of payouts - sum of costs
	Whichever combination gives the highest profit is the best solution.\\
	This gives a time complexity of $\mathcal(O)(n!)$, there surely must be better ways to approach this. We weren't able to come up with any though.\\
	
	The total profits can be calculated by adding up the profits and subtracting the costs. The profits is the total flow from $s$ to the $E_i$ edges and the cost is the total flow from $I_i$ to $t$.
	
	\subsection*{6.2}
	
	\begin{figure}[H]
		\includegraphics[scale=0.2]{/home/owner/git/algorithms/problems/week6/images/ford-faulkerson1.jpeg}
	\end{figure}

	\subsection*{6.3}
	
	To check whether its possible to connect each client to a base, we do the following:
	
	For each client, check to which possible bases they can connect according to the range limitation. Store the results in RESULT\_TABLE($\mathcal{O}(M^N))$.\\
	Make a graph, with a starting vertex $s$, and a sink $t$. Let clients and bases also be vertices. Connect $s$ to all of the clients, and let the cost be 1. Connect all bases to sink $t$, and 
	let the cost be the max load.\\
	Now connect clients to bases based on possible connections according to RESULT\_TABLE, with a cost of 1.\\
	Now run a max-flow algorithm like shortest-augmenting-path first ($\mathcal{O}(M^3)$). If the max-flow turns out to be equal to the number of clients, we can conclude that the answer to our main question is yes. If its lower, we can conclude the answer is no.
	
	\subsection*{6.4}
	
	False. In the following case it does not hold:
	
	\begin{figure}[H]
		\includegraphics[scale=0.2]{/home/owner/git/algorithms/problems/week6/images/examplegraph.jpeg}
	\end{figure}

	At first the min-cut is from S->A. When we add 1 to all the edges S->A is no longer a min-cut. Thus displacing the min-cut.

	\subsection*{6.5}
	
	Example:
	
	\begin{figure}[H]
		\includegraphics[scale=0.3]{/home/owner/git/algorithms/problems/week6/images/problem5.jpeg}
	\end{figure}

	The figure contains an example graph, the exact process for making the graph is as follows:\\
	
	- Draw a source vertex $s$ and a sink vertex $t$.\\
	- Add all the current users as a vertex $U_i$\\
	- Add all the different demographic groups as a vertex $G_i$\\
	- Add all the different advertisers as a vertex $M_i$\\
	- Let there be edges with cost 1 from $s$ to each $U_i$\\
	- Let there be edges from each $U_i$ to each $G_j$ if the user is part of that demographic group, let the cost be 1.\\
	- Let there be edges from each $G_j$ to each $M_k$ if the group $G_j$ is part of the audience of advertiser $M_k$, let the cost be infinite.\\
	- Let there be edges from each $M_k$ to the sink $t$, the cost here should be the demand of the advertiser $M_k$. (If $M_k$ wants to advertise to 10 people per minute, it should be 10).\\
	
	
	When we have the graph, the only thing we need to do is run a max-flow algorithm.\\
	Then we need to compare the max-flow to the sum of all the edges from M -> t, as those represent the demands of the different advertisers. If the max-flow is equal to that sum, then we can satisfy the demands of the advertisers. If no such flow exists, we won't be able to.
	
\end {document}